\documentclass[../main.tex]{subfiles}
 
\begin{document}

Begin the Introduction below the Keywords. The manuscript should not have headers, footers, or page numbers. It should be in a one-column format. References are often noted in the text and cited at the end of the paper.

\begin{table}[ht]
\caption{Fonts sizes to be used for various parts of the manuscript.  Table captions should be centered above the table.  When the caption is too long to fit on one line, it should be justified to the right and left margins of the body of the text.} 
\label{tab:fonts}
\begin{center}       
\begin{tabular}{|l|l|} %% this creates two columns
%% |l|l| to left justify each column entry
%% |c|c| to center each column entry
%% use of \rule[]{}{} below opens up each row
\hline
\rule[-1ex]{0pt}{3.5ex}  Article title & 16 pt., bold, centered  \\
\hline
\rule[-1ex]{0pt}{3.5ex}  Author names and affiliations & 12 pt., normal, centered   \\
\hline
\rule[-1ex]{0pt}{3.5ex}  Keywords & 10 pt., normal, left justified   \\
\hline
\rule[-1ex]{0pt}{3.5ex}  Abstract Title & 11 pt., bold, centered   \\
\hline
\rule[-1ex]{0pt}{3.5ex}  Abstract body text & 10 pt., normal, justified   \\
\hline
\rule[-1ex]{0pt}{3.5ex}  Section heading & 11 pt., bold, centered (all caps)  \\
\hline
\rule[-1ex]{0pt}{3.5ex}  Subsection heading & 11 pt., bold, left justified  \\
\hline
\rule[-1ex]{0pt}{3.5ex}  Sub-subsection heading & 10 pt., bold, left justified  \\
\hline
\rule[-1ex]{0pt}{3.5ex}  Normal text & 10 pt., normal, justified  \\
\hline
\rule[-1ex]{0pt}{3.5ex}  Figure and table captions & \, 9 pt., normal \\
\hline
\rule[-1ex]{0pt}{3.5ex}  Footnote & \, 9 pt., normal \\
\hline 
\rule[-1ex]{0pt}{3.5ex}  Reference Heading & 11 pt., bold, centered   \\
\hline
\rule[-1ex]{0pt}{3.5ex}  Reference Listing & 10 pt., normal, justified   \\
\hline
\end{tabular}
\end{center}
\end{table} 

\begin{table}[ht]
\caption{Margins and print area specifications.} 
\label{tab:Paper Margins}
\begin{center}       
\begin{tabular}{|l|l|l|} 
\hline
\rule[-1ex]{0pt}{3.5ex}  Margin & A4 & Letter  \\
\hline
\rule[-1ex]{0pt}{3.5ex}  Top margin & 2.54 cm & 1.0 in.   \\
\hline
\rule[-1ex]{0pt}{3.5ex}  Bottom margin & 4.94 cm & 1.25 in.  \\
\hline
\rule[-1ex]{0pt}{3.5ex}  Left, right margin & 1.925 cm & .875 in.  \\
\hline
\rule[-1ex]{0pt}{3.5ex}  Printable area & 17.15 x 22.23 cm & 6.75 x 8.75 in.  \\
\hline 
\end{tabular}
\end{center}
\end{table}

LaTeX margins are related to the document's paper size. The paper size is by default set to USA letter paper. To format a document for A4 paper, the first line of this LaTeX source file should be changed to \verb|\documentclass[a4paper]{spie}|.   

Authors are encouraged to follow the principles of sound technical writing, as described in Refs.~\citenum{Alred03} and \citenum{Perelman97}, for example.  Many aspects of technical writing are addressed in the {\em AIP Style Manual}, published by the American Institute of Physics.  It is available on line at \url{https://publishing.aip.org/authors}. A spelling checker is helpful for finding misspelled words. 

An author may use this LaTeX source file as a template by substituting his/her own text in each field.  This document is not meant to be a complete guide on how to use LaTeX.  For that, please see the list of references at \url{http://latex-project.org/guides/} and for an online introduction to LaTeX please see \citenum{Lees-Miller-LaTeX-course-1}. 

\end{document}