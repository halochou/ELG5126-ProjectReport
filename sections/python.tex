\documentclass[../main.tex]{subfiles}
 
\begin{document}
We chose Python as the main programming language to implement this project.

Python is a widely used high-level, general-purpose, interpreted, dynamic programming language. 
Its design philosophy emphasizes code readability, 
and its syntax allows programmers to express concepts in fewer lines of code than possible in languages such as C++ or Java. 
\footnote{From Wikipedia: https://en.wikipedia.org/wiki/Python\_(programming\_language)}

Although programs written in C/C++ mostly have significantly better performance then ones in Python, they have to pay for verbosity as trade-off.
Due to the statement from Python's official website, Python code is typically 5-10 times shorter than equivalent C++ code.
\footnote{From Python official website: https://www.python.org/doc/essays/comparisons}

The dynamic property and modern syntax of Python can eliminate much more noise in the code, making the idea behind the program more clear. For example, to calculate Fibonacci sequence in plain iterative method, we can implement it as below:

C:
\begin{verbatim}
    long long int fibb(int n) {
        int fnow = 0, fnext = 1, tempf;
        while(--n>0){
            tempf = fnow + fnext;
            fnow = fnext;
            fnext = tempf;
            }
            return fnext;   
    }
\end{verbatim}

Python:
\begin{verbatim}
    def fibIter(n):
        if n < 2:
            return n
        fibPrev = 1
        fib = 1
        for num in xrange(2, n):
            fibPrev, fib = fib, fib + fibPrev
        return fib
\end{verbatim}

As a project which aims to be a concept prototype, we care more about readability and productivity instead of performance. Python works well for this target.

\end{document}