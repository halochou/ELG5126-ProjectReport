\documentclass[../main.tex]{subfiles}
 
\begin{document}

Figures are numbered in the order of their first citation.  They should appear in numerical order and on or after the same page as their first reference in the text.  Alternatively, all figures may be placed at the end of the manuscript, that is, after the Reference section.  It is preferable to have figures appear at the top or bottom of the page.  Figures, along with their captions, should be separated from the main text by at least 0.2 in.\ or 5 mm.  

Figure captions are centered below the figure or graph.  Figure captions start with the figure number in 9-point bold font, followed by a period; the text is in 9-point normal font; for example, ``{\footnotesize{Figure 3.}  Original image...}''.  See Fig.~\ref{fig:example} for an example of a figure caption.  When the caption is too long to fit on one line, it should be justified to the right and left margins of the body of the text.  

Tables are handled identically to figures, except that their captions appear above the table. 

   \begin{figure} [ht]
   \begin{center}
   \begin{tabular}{c} %% tabular useful for creating an array of images 
   \includegraphics[height=5cm]{mcr3b.eps}
   \end{tabular}
   \end{center}
   \caption[example] 
%>>>> use \label inside caption to get Fig. number with \ref{}
   { \label{fig:example} 
Figure captions are used to describe the figure and help the reader understand it's significance.  The caption should be centered underneath the figure and set in 9-point font.  It is preferable for figures and tables to be placed at the top or bottom of the page. LaTeX tends to adhere to this standard.}
   \end{figure} 
   
\end{document}