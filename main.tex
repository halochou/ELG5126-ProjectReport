\documentclass[]{spie}  %>>> use for US letter paper
%\documentclass[a4paper]{spie}  %>>> use this instead for A4 paper
%\documentclass[nocompress]{spie}  %>>> to avoid compression of citations

\renewcommand{\baselinestretch}{1.0} % Change to 1.65 for double spacing
 
\usepackage{amsmath,amsfonts,amssymb}
\usepackage{graphicx}
\graphicspath{{images/}{../images/}}

\usepackage{subfiles}
\usepackage[colorlinks=true, allcolors=blue]{hyperref}

\title{H.264 / MPEG 4 AVC Decoder Implementation}

\author[a]{Fanyu Ran}
\author[b]{Yang Zhou}
\author[c]{Yifei Zhou}
\affil[a]{Student No.8657223, University of Ottawa}
\affil[b]{Student No.8657223, University of Ottawa}
\affil[c]{Student No.8657223, University of Ottawa}

\authorinfo{Further author information:\\Fanyu Ran: E-mail: aaa@tbk2.edu \\ Yang Zhou: E-mail: bba@cmp.com \\ Yifei Zhou: E-mail: bba@cmp.com}

% Option to view page numbers
\pagestyle{empty} % change to \pagestyle{plain} for page numbers   
\setcounter{page}{301} % Set start page numbering at e.g. 301
 
\begin{document} 
\maketitle

\begin{abstract}
This document is prepared using LaTeX2e\cite{Lamport94} and shows the desired format and appearance of a manuscript prepared for the Proceedings of the SPIE.\footnote{The basic format was developed in 1995 by Rick Hermann (SPIE) and Ken Hanson (Los Alamos National Lab.).} It contains general formatting instructions and hints about how to use LaTeX.  The LaTeX source file that produced this document, {\ttfamily article.tex} (Version 3.4), provides a template, used in conjunction with {\ttfamily spie.cls} (Version 3.4). These files are available on the Internet at \url{https://www.overleaf.com}.  The font used throughout is the LaTeX default font, Computer Modern Roman, which is equivalent to the Times Roman font available on many systems.  
\end{abstract}

% Include a list of keywords after the abstract 
\keywords{Manuscript format, template, SPIE Proceedings, LaTeX}

\section{INTRODUCTION}
\label{sec:intro}  % \label{} allows reference to this section
\subfile{sections/introduction}

\section{Python Programming Language Overview}
\label{sec:python}  % \label{} allows reference to this section
\subfile{sections/python}

\section{H.264 Decoding Overview}
\label{sec:decoding}
\subfile{sections/decoding}

\section{Software Design Overview}
\label{sec:design}
\subfile{sections/design}

\section{H.264 and Implementation Details}
\label{sec:implementation}
\subfile{sections/implementation}

\section{Result and Design Exploration}
\label{sec:result}
\subfile{sections/result}

\appendix    %>>>> this command starts appendixes

\section{MISCELLANEOUS FORMATTING DETAILS}
\label{sec:misc}
\subfile{sections/appendix}

\acknowledgments % equivalent to \section*{ACKNOWLEDGMENTS}       
\label{sec:acknowledgments}
\subfile{sections/acknowledgments}

% References
\bibliography{reference} % bibliography data in report.bib
\bibliographystyle{spiebib} % makes bibtex use spiebib.bst

\end{document} 
